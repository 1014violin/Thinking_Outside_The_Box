\documentclass[11pt, twoside, reqno]{book}
\usepackage{amssymb, amsthm, amsmath, amsfonts}
\usepackage{graphicx}
\usepackage{amsrefs}
\usepackage{color}
\usepackage{hyperref}
\usepackage{bardtex}
\usepackage{adjustbox}
\usepackage{fancyvrb}
\usepackage{pdfpages}
\usepackage{placeins}
\usepackage{float}




\newcommand{\cpa}{\text{chessboard pixel area}}

\styleoption{seniorproject}

%Your macros, if you have any.


\begin{document}

\titlepg{Thinking Outside The Box: \\Computing 3D Volume in 2D}{Alexandra Morris}
    {December}{2017}

\abstr


This project explores how to compute 3D volume of cardboard boxes in 2D without a calibrated camera. Computer vision techniques to obtain 3D volume typically require camera calibration, the standard method for mapping 3D points to 2D.
We created our own solution that doesn't rely on camera calibration and obtains the areas of each box with unknown dimensions with the help of a chessboard pattern placed on each box side. The solution is a proportion that given the box area in pixels, chessboard pattern in pixels, and the chessboard pattern in inches, determines the box area in inches. We tested this method on 20 boxes, 5 pictures of each side for one box. The results showed positive feedback compared with the defined areas/ volumes and compared with the results of our Homographies. Ultimately we determined that our solution has the potential, with improved photos, test methods, etc. to accurately find an unknown box's volume given only the provided 2D data.

\tableofcontents

\dedic

I would like to dedicate my research to my parents who have relentlessly cultivated my curiosity and guided me to be disciplined in order to achieve my goals. In addition I dedicate my project to my sister Avery, who is the best sister that anyone could ask for. In addition, I thank my Grandma, Aunt Frannie and Uncle Richard, for their unrelenting generosity and encouragement. Without them, this project might not have happened. 

\acknowl

I thank and acknowledge Keith for guiding me along the way and initiating and cultivating my fascination with computer vision. I am indebted to his support, in addition to the support from the other computer science faculty, Sven and Becky who similarly have been profound inspirations in my life. 

\startmain


\chapter{Introduction}
\label{label}

\section{Motivation}
\label{label}

You are taping up boxes of winter gear to put away in storage. In anticipation of having more than the average amount of boxes, you call the storage facility and request a large unit. Now that the boxes are all packed up, you load them into the car, and drive to the storage unit. After unloading and placing the boxes in the unit, you realize you've made a terrible mistake; you've rented out a unit twice the size of all your boxes. 
Precision in the process of moving, whether it involves keeping track of your belongings in each box or knowing the total amount of goods you are storing, is a field that lags behind in fulfilling modern technology's potential. At the end of each school year, Bard students face this dilemma with regards to moving. Bard College requires that each student move his/her belongings out of his/her dorm for the summer. Each year, students become over sized storage unit victims and overpay for something that should be so simple.
Through this study, we investigate how to obtain a box's volume (with unknown dimensions) without using camera calibration, which is the standard method for mapping 3D points to 2D.\cite{Solem} Not only is obtaining box size data important for individuals planning on moving or renting a storage unit, but imagine the negative financial repercussions of a UPS employee who isn't sure how many boxes he/she can fit in his truck? Envision a company that manages a large amount of inventory overpaying for an over sized space thus setting them back financially? In addition, according to the United States Census Bureau, young adults have the highest migration rate compared to any other generation and account for 43\% of movers\cite{census}. Due to this statistic, the Millennial generation, which relies heavily on smart phone assistance would benefit greatly from a program to detect box sizes in order to improve their moving experiences. Gaining 3D data (without relying on camera calibration) for any box size is not only a fascinating computer vision problem, but if well executed, there could be an economic demand to solve this problem for the Millennial generation. In this project, we ask: Can we accurately obtain a box's volume of unknown dimensions using an uncalibrated camera? We created our own solution that doesn't rely on camera calibration and obtains the areas of each box of unknown dimensions with the help of chessboard pattern placed on each box side. The solution is a proportion that given the box area in pixels, chessboard pattern in pixels, and the chessboard pattern in inches, determines the box area in inches. We tested this method on 20 boxes, 5 pictures of each side for one box and obtained positive results that were similar to our defined data.



\chapter{Background}
\label{label}

\section{Pin-Hole Cameras}
\label{label}
\subsection{Homogeneous Coordinates}
 Homogeneous coordinates are essentially cartesian coordinates $ \begin{bmatrix}
 x, y
 \end{bmatrix}$ with the addition of another parameter: w, or the scalar. 

\begin{center}
	
	$\widetilde{v} =
	\begin{bmatrix}
	x\\
	y\\
	w\\
	\end{bmatrix}$
	
\end{center}
 
 Homogeneous coordinates are widespread within computer vision as they enable transformations such as translation, rotation, scaling and perspective projection to be represented as a matrix by which the homogeneous coordinate vector is multiplied. Homogeneous coordinates can be used for both 2D and 3D points making them incredibly versatile\cite{sze}. 
 
 \begin{center}
	$\widetilde{v} =
\begin{bmatrix}
x\\
y\\
z\\
w\\
\end{bmatrix}$

 \end{center}

 In addition, it is very simple to convert homogeneous coordinates to non-homogeneous coordinates. All you need to do is divide all parameters by the scalar, w\cite{sze}. 
 
\begin{center}
	
	$\widetilde{v} =
	\begin{bmatrix}
	x\\
	y\\
	w\\
	\end{bmatrix} \,\to\, \begin{bmatrix}
	x/w\\
	y/w\\

	\end{bmatrix} = v $
	
\end{center}
 
 If you have a non-homogeneous point and want it to be homogeneous, all you need to do is add the additional scale parameter, and normalize it making $w = 1$. \cite{sze} 
 
 \begin{center}
 	
 	$v =
 	\begin{bmatrix}
 	x\\
 	y\\
 	\end{bmatrix}  \,\to\,
 	\begin{bmatrix}
 	x\\
 	y\\
 	1\\
 	\end{bmatrix} =\widetilde{v}$
 	
 \end{center}

\subsection{Transformations}

Homogeneous coordinates are essential to completing nearly all computer vision geometric transformations that account for perspective. But first, it is necessary to understand the pin-hole camera model.

Much of computer vision assumes the pin-hole camera model (also known as camera obscura). Solem writes about the pin-hole camera in \textbf{Programming Computer Vision with Python:}

``The name comes from the type of camera, like  a camera obscura, that collects light through a small hole to the inside of a dark box or room. In the pin-hole camera model, light passes through a single point, the camera center, C before it is projected in front of the camera center. The image plane in an actual camera would be upside down behind the camera center, but the model is the same\cite{Solem}.'' 

A camera obscura is in its basic form a box with a hole on one side. Light passing through that hole forms an inverted image of the object on the opposite side of the box~\cite{spixel}.

\FloatBarrier
	
	\begin{figure}
	\graphicspath{ {images/} }
	\begin{center}
	\includegraphics [scale=2]{cameraobscura}
	\caption{A camera obscura model  ~\cite{spixel}}	
	\end{center}
	\end{figure}
	



\begin{figure}
	\graphicspath{ {images/} }
	\includegraphics{pinhole_camera}
	\caption{The pin-hole camera model has defined variables enabling a variety of calculations 	\cite{Solem}}

\end{figure}


The pin-hole camera model, a more sophisticated camera obscura defines variables used for transformations. In this model, C represents the camera center and $f$ is the focal length or distance from the camera center to the image plane. The image point, $c = [c_x,c_y]$ is where the optical axis intersects the image plane\cite{Solem}. X is the 3D point, and x is the 2D image point. 




We can use the pin-hole camera model to perform transformations allowing us to manipulate images. With the additional help of homogeneous coordinates, we can perform translation, rotation, scaling, perspective projections, and calibrate cameras. 

\begin{figure}
	\graphicspath{ {images/} }
	
	\includegraphics{transforms}
	\label{transforms}
	\caption{Homogeneous coordinates enable the included transformations \cite{sze}}
	
\end{figure}
\FloatBarrier


\subsection{Camera Calibration}
We can model the pin-hole camera by a matrix that maps 3D points to 2D\cite{Solem}. We first find the camera matrix, which  includes many of the same variables from the Pin-Hole camera model. The intrinsics matrix (K) includes the focal length and the image point (c). 
\begin{center}
	
	$K =
	\begin{bmatrix}
	f_x&0&c_x\\
	0&f_y&c_y\\
	0&0&1\\
	\end{bmatrix}$
	
\end{center}

Next the camera matrix (P) is obtained when the intrinsics matrix is multiplied by the extrinsics matrix $[R \mid \textbf{t}]
$. 
\[
P=K[R \mid \textbf{t}]
\]
Finally, the camera matrix is multiplied by the 3D point (X) to determine the 2D 
point (x). 

\[
x=PX
\]

The calibration matrix is composed of translation and rotation transformations. Camera calibration is the process of finding P (the camera matrix) from 3D/2D corresponding coordinates, though for this study, it is unnecessary and we work only with 2D planes. 


\subsection{Homographies}

\begin{figure}[H]
	\begin{center}
		\graphicspath{ {images/} }
		\includegraphics[scale=.5]{local-homographies} 
		\caption{Homographies map points from one plane to another  \cite{homography}}
	\end{center}
\end{figure}

 Homographies are essentially projective transformations. See figure \ref{transforms}  
 They are used to map points from one projective space to another\cite{Solem}. The concept of mapping one projective space to another can be seen in this image. 
 

We used the OpenCV library's function \texttt{getPerspectiveTransform()} in order to obtain our homography data. 



\section{Resources}
\label{label}
\subsection{OpenCV}
We used the OpenCV library as the basis of our project. The OpenCV Library, or Open Computer Vision Library is an open source computer vision and machine learning library. It contains more than 2500 optimized algorithms ranging from face detection, to camera calibration, to augmented reality. This Library can be used in Python, C++, and Java\cite{opencvCC}.  
Due to the simple interface, we decided to use Processing, a more user-friendly, graphically minded Java platform. We began using the data structures \texttt{PVectors} and \texttt{Mat}, and the OpenCV functions \texttt{findChessboardCorners()} and \texttt{getPerspectiveTransform()}. 



\texttt{PVectors} are essentially 2 or 3 dimension vectors that take both magnitude and direction as inputs. The \texttt{PVector class} has convenient functions that allow obtaining magnitude, direction, multiplying or adding vectors simple. We incorporated \texttt{PVectors} in order to acquire both the box and chessboard areas, enabling us to accurately calculate the area of a box side in pixels. Though we used \texttt{findChessboardCorners()} for a majority of the study, we ended up not using it as it presented too many bugs to enhance our program. 

\texttt{findChessboardCorners()} takes a grid chessboard image and determines the 2D pixel locations of the corners. 

\chapter{Methods and Execution}
\label{label}

\subsection{Proposed Algorithm for finding area} 

\begin{figure} [H]
	\graphicspath{ {images/} }
	\includegraphics[scale=.45]{box}
	\caption{A box with planes A, B, and labeled}
\end{figure}

	


 \textbf{ 1. Load box side image.} \\
The box sides are categorized as the following:\\
\[Plane\ A = l*w\] 
\[Plane\ B = l*h\] 
\[Plane\ C = w*h\] 
Each image is identified as: BoxIdentifier\_PlaneType\_PictureNumber.jpg \\

\textbf{2. Enter the number of chessboard squares in pop-up window.}\\
The chessboard number only includes squares that have a clear boundary\ edge.The user clicks on the 4 corners (clockwise, starting from top left) that make up the perimeter of the chessboard region. \\

\begin{figure}[H]
	\begin{center}
				\graphicspath{ {images/} }
		
		\includegraphics[scale=.05]{opaque}
		\caption{Chessboard area region}
	\end{center}

\end{figure}



\textbf{3. The user clicks on the 4 corners of the box plane region, clockwise, starting from top left.}\\

\begin{figure} [H]
	\begin{center}
			\graphicspath{ {images/} }
		\includegraphics[scale=.2]{clicks_plane}
		\caption{Box side area region}
	\end{center}
	
\end{figure}


\textbf{4. Compute the real area (in inches) of the chessboard region.}
 \[chessRealArea = .828*box\_num\] 
 \begin{center}
.828 is the area of one single chessboard square.
 \end{center} 
 
  \textbf{5. The areas (in pixels) of the box side and chessboard region are calculated}
 
 
  \textbf{6. The real area (in inches) is found using the following proportion}
  \[ \frac{Chessboard\ Region\ Area\ (PX)}{Box\ Plane\ Region\ (PX)} = \frac{Chessboard\ Region\ Area\ (IN)}{Box\ Plane\ Region\ (IN)}  \]
   \[ {Box\ Plane\ Region\ (IN)} = \frac{Box\ Plane\ Region\ (PX)*Chessboard\ Region\ Area\ (IN)}{Chessboard\ region\ area\ (PX)}  \]


The user defines the chessboard region to have 24 squares as sectioned off in the above image (for the majority of the images). The corners of the box side region are shown in blue. Our area finding algorithm is repeated for each of the three planes.

To find the pixel areas, we needed to ensure that our algorithm was adaptable for any quadrilateral, as the pixel areas are not specifically defined quadrilaterals as they would be in real world areas. Given that we had 4 PVectors (each being one of the clicked corners), we found two parallel angles within the quadrilateral. Then, we extracted the lengths of all the sides, and calculated the following area:


\[area = .5*(a*b)*\sin(ang\_AB) + .5*(c*d)*\sin(ang\_CD)\] 
\[a\ ,b\ ,c\ ,d\ \textrm{are\ the\ distances\ between\ corners}\]
\[ang\_AB = \textrm{the\ angle\ in\ between\ distance\ a\ and\ b}\]
\[ang\_CD = \textrm{the\ angle\ in\ between\ distance\ c\ and\ d}\]

 While each side's area is being found, the length, width, and height of the box (in inches) are not able to be obtained as our ratio only relates area, not side measurements. Thus, we needed a way to determine the volume of a box with only the three given areas. Using substitution, we found a succinct formula for the box's volume in inches. 

Given that-
\[Plane\ A = l*w\]
\[Plane\ B = h*l\]
\[Plane\ C = h*w\]


from the formula, \[V = l*w*h\] the volume of a box is \[Volume =Plane\ B*\sqrt{\frac{Plane\ C*Plane\ A}{Plane\ B}} \].

By substitution:

\[V = l*w*h\]
\[V = B*w\]
and,
\[A = l*w\]
\[l =\frac{A}{w}\]
Then,
\[B = h*l\]
\[B = h*\frac{A}{w}\]
\[\frac{w}{A}*B = h*\frac{A}{w}*\frac{w}{A}\]
\[\frac{w}{A}*B = h\]
Similarly,
\[C = h*w\]
\[h = \frac{C}{w}\]
Next,
\[\frac{w}{A}*B = h\]
\[\frac{w}{A}*B = \frac{C}{w}\]
\[\frac{A}{w}*\frac{w}{A}*B = \frac{C}{w}*\frac{A}{w}\]
\[B = \frac{C*A}{w^2}\]
\[\frac{1}{C*A}*B = \frac{C*A}{w^2}*\frac{1}{C*A}\]
\[\frac{B}{C*A} = \frac{1}{w^2}\]
\[B*w^2= C*A\]
\[\frac{B*w^2}{B} = \frac{C*A}{B}\]
\[\sqrt{w^2} = \sqrt{\frac{C*A}{B}}\]
\[w = \sqrt{\frac{C*A}{B}}\]
Finally,
\[V= l*w*h\] 
\[V= B*w \]
\[Volume =Plane\ B*\sqrt{\frac{Plane\ C*Plane\ A}{Plane\ B}} \].



\subsection{Homographies}



The homography method used to verify our algorithm has two components. First, we find the homography that maps the chessboard pixel corners to the chessboard inch corners. Since the chessboard squares each have a pre-measured side length, we could easily obtain the chessboard coordinates in inches. \\ The built in OpenCV homography function called \texttt{getPerspectiveTransform()} accepts the 4 outer chessboard pixel coordinates as matrices. This function takes two matrices the source and the destination. It returns the 3x3 homography matrix. Still using homogeneous coordinates, we need to relate the resulting homography (H) to the box area pixel corners. In order to do this, we take our 3x4 matrix of all the box area pixel corners and use matrix multiplication to merge it with our H. Finally we convert the resulting 3x4 matrix to non-homogeneous coordinates and obtain the real area of each side in inches. 

The homography algorithm to find the real areas (in inches) works as follows:
\\ \\
1. We input the total number of chessboard squares, and pre-define the chessboard square length thus the chessboard area is determined. \\ \\
2. The user inputs: \\ 
-the chessboard area pixel coordinates by clicking on the corners. \\
-the box area pixel coordinates by also selecting the corners.\\ \\
3. The OpenCV function \texttt{getPerspectiveTransform()} is called and returns the homograpy matrix. \\ \\
4. The homography matrix is multiplied by the box area pixel coordinates to obtain the real area in inches. 
\chapter{Results}
\label{label}

\subsection{Algorithmic Results}
	For the majority of our tests, we used International Paper boxes, a box company that Amazon commonly uses to ship their products. International Paper boxes were a convenient option as each box had a labeled identifier (e.g 1A7 or 1AD) and the dimensions of these boxes were easy to obtain online~\cite{intpaper}. Soon after evaluating the pre-measured International Paper box dimensions, we determined that the length, width, and height of the boxes are measured when the box is flat packed. Thus for the other boxes which we measured by hand, we made the necessary adjustments to have the measurements consistent with International Paper's measurements. 

\begin{figure} [H]
	\begin{center}
		\graphicspath{ {images/} }
		
		\includegraphics[scale=.5]{flat}
		\caption{A Flat-packed box.~\cite{intpaper}} 
		
	\end{center}
	
	
\end{figure}





\begin{figure} [H]
	\graphicspath{ {images/} }
	\includegraphics{box_range}
	\caption{Test box size range \cite{intpaper}}

\end{figure}

	
	The equipment used to obtain our results included an iPhone camera, 20 brown cardboard boxes (International Paper Brand, Target, Lowes, etc.), a printed copy of OpenCV chessboard pattern, packing tape, and a gray backdrop (which provided enough contrast to the brown boxes)~\cite{gregProc}~\cite{intpaper}. \\
	For our volume algorithm tests, we had the following assumptions: \\
	\\
	1. There are chessboard squares in a grid on each box surface.\\
	\\
	2. The box surface is flat and parallel to the chessboard square sheet of paper.\\
	\\
	3. All box side corners are visible.  \\
	\\
	These are the following types of images in our dataset: \\ \\
	\FloatBarrier
	\graphicspath{ {images/} }
	1. The image plane is parallel to the 3D points and there are 24 chessboard squares counted.  \\ \\
	\includegraphics[scale=.05]{1_A_4}
	\\ 
	
	\graphicspath{ {images/} }
	2. The image plane is parallel to 3D point and less chessboard squares are on the box sides. \\ \\
	\includegraphics[scale=.05]{4_C_4} \\
	
	\graphicspath{ {images/} } 
	
	3. The smaller box surface and chessboard square pattern extend beyond the side's area.\\ \\
	\includegraphics[scale=.05]{2_C_2} \\ 
	
	\graphicspath{ {images/} }
	4. 24 square image with distortion \\ \\
	\includegraphics[scale=.05]{5_A_5}\\ 
	
	\graphicspath{ {images/} }
	5. The image contains less than 24 squares and the image is distorted. \\ \\
	\includegraphics[scale=.05]{4_C_2}  \\ 


The boxes we tested ranged in volume between $160 in^3$ to $6672.75 in^3$. We tested our algorithm on 20 boxes, 5 images per box side, thus 15 images per box. The results and data extracted from these tests are listed below. 
\FloatBarrier
\begin{table}[H]
	\centering

	\label{AR2E}
	\begin{tabular}{llll}
		Box                 & Avg Volume & Defined Volume & Volume Avg Error \\
		A3 or BY1           & 383.07     & 367.50         & 4.24\%           \\
		Shar                & 153.15     & 224.86         & 31.89\%          \\
		int paper 1AD BNA   & 280.70     & 448.88         & 37.47\%          \\
		LLBean              & 330.77     & 550.00         & 39.86\%          \\
		Target model 439 sm & 1385.50    & 1653.75        & 16.22\%          \\
		N3 or B41           & 617.94     & 1023.75        & 39.64\%          \\
		CVS box             & 2383.09    & 3655.39        & 34.81\%          \\
		Lowes Box           & 1824.99    & 2304.00        & 20.79\%          \\
		Int Paper 1A5 B45   & 596.10     & 705.38         & 15.49\%          \\
		int paper 0A0       & 94.61      & 135.28         & 30.06\%          \\
		Int paper A4        & 550.63     & 714.00         & 22.88\%          \\
		Int paper 1A7       & 702.78     & 841.00         & 16.44\%          \\
		int paper BP0       & 122.75     & 160.00         & 23.28\%          \\
		int paper 2AA       & 1356.58    & 1632.00        & 16.88\%          \\
		int paper 2A0       & 1873.55    & 2240.00        & 16.36\%          \\
		int paper BP1       & 226.97     & 283.22         & 19.86\%          \\
		int paper 2A5       & 785.14     & 935.00         & 16.03\%          \\
		int paper 2A8       & 3822.86    & 4446.00        & 14.02\%          \\
		int paper 3A1       & 6652.82    & 6672.75        & 0.30\%           \\
		int paper B0        & 1820.16    & 2057.00        & 11.51\%         
	\end{tabular}
\caption{Volume results, consisting of average A, B, and C box side values}
\end{table}
\FloatBarrier

\begin{table}[]
	\centering
	
	\label{my-label}
	\begin{tabular}{lllllll}
		Box                 & length & width & height & Defined Area (A) & Defined Area (B) & Defined Area (C) \\
		A3 or BY1           & 10     & 7     & 5.25   & 70.00            & 52.5             & 36.75            \\
		Shar                & 10.25  & 6.75  & 3.25   & 69.19            & 33.3125          & 21.9375          \\
		int paper 1AD BNA   & 13.5   & 9.5   & 3.5    & 128.25           & 47.25            & 33.25            \\
		LLBean              & 13.75  & 8     & 5      & 110.00           & 68.75            & 40               \\
		Target model 439 sm & 17.5   & 13.5  & 7      & 236.25           & 122.5            & 94.5             \\
		N3 or B41           & 16.25  & 12    & 5.25   & 195.00           & 85.3125          & 63               \\
		CVS box             & 21.65  & 13.4  & 12.6   & 290.11           & 272.79           & 168.84           \\
		Lowes Box           & 16     & 12    & 12     & 192.00           & 192              & 144              \\
		Int Paper 1A5 B45   & 13.5   & 11    & 4.75   & 148.50           & 64.125           & 52.25            \\
		int paper 0A0       & 9.25   & 6.5   & 2.25   & 60.13            & 20.8125          & 14.625           \\
		Int paper A4        & 12     & 8.5   & 7      & 102.00           & 84               & 59.5             \\
		Int paper 1A7       & 14.5   & 8     & 7.25   & 116.00           & 105.125          & 58               \\
		int paper BP0       & 10     & 8     & 2      & 80.00            & 20               & 16               \\
		int paper 2AA       & 24     & 16    & 4.25   & 384.00           & 102              & 68               \\
		int paper 2A0       & 20     & 16    & 7      & 320.00           & 140              & 112              \\
		int paper BP1       & 13.25  & 9.5   & 2.25   & 125.88           & 29.8125          & 21.375           \\
		int paper 2A5       & 20     & 11    & 4.25   & 220.00           & 85               & 46.75            \\
		int paper 2A8       & 26     & 19    & 9      & 494.00           & 234              & 171              \\
		int paper 3A1       & 31     & 20.5  & 10.5   & 635.50           & 325.5            & 215.25           \\
		int paper B0        & 17     & 11    & 11     & 187.00           & 187              & 121             
	\end{tabular}
\caption{Pre-measured areas for all boxes}
\end{table}

\FloatBarrier

\begin{table}[]
	\centering

	\label{AR2B}
	\tiny
	\begin{tabular}{lllllll}
		Box                 & (A) Min Error & (B) Min error & (C) Min Error & (A) Max Error & (B) Max Error & (C) Max Error \\
		A3 or BY1           & 8.05\%        & 0.40\%        & 5.53\%        & 13.29\%       & 5.45\%        & 34.87\%       \\
		Shar                & 0.50\%        & 9.09\%        & 6.38\%        & 4.05\%        & 15.17\%       & 18.97\%       \\
		int paper 1AD BNA   & 0.89\%        & 10.12\%       & 1.14\%        & 16.21\%       & 14.74\%       & 9.25\%        \\
		LLBean              & 0.73\%        & 7.29\%        & 2.70\%        & 8.44\%        & 16.20\%       & 12.65\%       \\
		Target model 439 sm & 0.18\%        & 17.98\%       & 9.13\%        & 10.82\%       & 36.59\%       & 23.82\%       \\
		N3 or B41           & 0.77\%        & 0.28\%        & 0.84\%        & 4.47\%        & 13.49\%       & 43.42\%       \\
		CVS box             & 0.40\%        & 1.36\%        & 3.04\%        & 7.12\%        & 9.17\%        & 9.93\%        \\
		Lowes Box           & 6.09\%        & 6.66\%        & 10.56\%       & 10.70\%       & 10.70\%       & 21.04\%       \\
		Int Paper 1A5 B45   & 9.81\%        & 16.32\%       & 8.61\%        & 16.16\%       & 23.16\%       & 16.83\%       \\
		int paper 0A0       & 0.67\%        & 0.91\%        & 3.06\%        & 13.23\%       & 7.41\%        & 14.75\%       \\
		Int paper A4        & 11.28\%       & 1.13\%        & 6.80\%        & 13.80\%       & 11.88\%       & 12.98\%       \\
		Int paper 1A7       & 18.18\%       & 6.98\%        & 10.18\%       & 23.43\%       & 10.00\%       & 17.30\%       \\
		int paper BP0       & 10.00\%       & 1.10\%        & 0.51\%        & 19.68\%       & 4.68\%        & 20.29\%       \\
		int paper 2AA       & 13.97\%       & 7.51\%        & 13.20\%       & 15.96\%       & 12.93\%       & 21.32\%       \\
		int paper 2A0       & 15.68\%       & 8.97\%        & 13.50\%       & 17.04\%       & 14.20\%       & 18.89\%       \\
		int paper BP1       & 13.09\%       & 4.33\%        & 2.54\%        & 18.72\%       & 12.20\%       & 20.64\%       \\
		int paper 2A5       & 11.21\%       & 11.23\%       & 10.68\%       & 14.35\%       & 20.39\%       & 21.69\%       \\
		int paper 2A8       & 13.05\%       & 12.96\%       & 15.75\%       & 15.66\%       & 19.58\%       & 22.09\%       \\
		int paper 3A1       & 13.74\%       & 50.64\%       & 18.68\%       & 17.77\%       & 55.36\%       & 23.15\%       \\
		int paper B0        & 11.02\%       & 18.78\%       & 20.50\%       & 19.35\%       & 21.36\%       & 24.77\%      
	\end{tabular}
\caption{Minimum and maximum error percentages for box side A, B, and C}
\end{table}

\FloatBarrier
\begin{table}[]
	\centering

	\label{AR2C}
	\begin{tabular}{llll}
		Box                 & (A) Avg Error & (B) Avg Error & (C) Avg Error \\
		A3 or BY1           & 4.42\%        & 2.36\%        & 16.43\%       \\
		Shar                & 0.83\%        & 10.70\%       & 9.29\%        \\
		int paper 1AD BNA   & 5.09\%        & 12.77\%       & 1.40\%        \\
		LLBean              & 3.82\%        & 12.13\%       & 8.14\%        \\
		Target model 439 sm & 1.81\%        & 24.95\%       & 18.44\%       \\
		N3 or B41           & 2.41\%        & 4.51\%        & 16.08\%       \\
		CVS box             & 2.73\%        & 0.57\%        & 5.67\%        \\
		Lowes Box           & 8.25\%        & 7.91\%        & 15.30\%       \\
		Int Paper 1A5 B45   & 13.58\%       & 19.87\%       & 12.61\%       \\
		int paper 0A0       & 5.18\%        & 1.72\%        & 8.86\%        \\
		Int paper A4        & 12.54\%       & 4.27\%        & 8.81\%        \\
		Int paper 1A7       & 21.31\%       & 8.57\%        & 13.81\%       \\
		int paper BP0       & 13.92\%       & 0.16\%        & 10.73\%       \\
		int paper 2AA       & 14.86\%       & 10.84\%       & 16.51\%       \\
		int paper 2A0       & 16.43\%       & 11.29\%       & 15.89\%       \\
		int paper BP1       & 15.76\%       & 8.32\%        & 9.95\%        \\
		int paper 2A5       & 12.63\%       & 15.09\%       & 16.78\%       \\
		int paper 2A8       & 14.48\%       & 16.39\%       & 19.12\%       \\
		int paper 3A1       & 15.51\%       & 52.24\%       & 21.34\%       \\
		int paper B0        & 14.58\%       & 19.80\%       & 22.46\%      
	
\end{tabular}
\caption{Average errors for each box side}
\end{table}


\FloatBarrier

\begin{table}[]
	\centering

	\label{AR2Dl}
	\begin{tabular}{lllllll}
		Box                 & Min A  & Max A  & Min B  & Max B  & Min C  & Max C  \\
		A3 or BY1           & 63.49  & 79.30  & 49.64  & 53.81  & 38.78  & 49.57  \\
		Shar                & 66.38  & 71.52  & 36.34  & 38.36  & 17.78  & 23.40  \\
		int paper 1AD BNA   & 107.46 & 130.52 & 40.29  & 42.47  & 32.35  & 36.33  \\
		LLBean              & 100.72 & 109.20 & 57.61  & 63.74  & 34.94  & 38.92  \\
		Target model 439 sm & 223.35 & 261.82 & 144.53 & 167.32 & 103.13 & 117.01 \\
		N3 or B41           & 186.28 & 196.50 & 74.32  & 96.82  & 35.64  & 65.29  \\
		CVS box             & 269.46 & 304.77 & 256.04 & 297.82 & 152.08 & 163.71 \\
		Lowes Box           & 203.69 & 212.54 & 204.79 & 212.55 & 159.21 & 174.29 \\
		Int Paper 1A5 B45   & 163.07 & 172.50 & 74.59  & 78.97  & 56.75  & 61.04  \\
		int paper 0A0       & 52.17  & 60.53  & 20.46  & 22.36  & 14.18  & 16.78  \\
		Int paper A4        & 113.51 & 116.07 & 82.99  & 93.98  & 63.55  & 67.22  \\
		Int paper 1A7       & 137.09 & 143.18 & 112.46 & 115.64 & 63.91  & 68.03  \\
		int paper BP0       & 88.00  & 95.74  & 19.34  & 20.94  & 15.92  & 19.25  \\
		int paper 2AA       & 437.66 & 445.27 & 109.66 & 115.18 & 76.97  & 82.50  \\
		int paper 2A0       & 370.17 & 374.53 & 152.56 & 159.88 & 127.12 & 133.15 \\
		int paper BP1       & 142.35 & 149.43 & 31.10  & 33.45  & 21.92  & 25.79  \\
		int paper 2A5       & 244.66 & 251.56 & 94.54  & 102.34 & 51.74  & 56.89  \\
		int paper 2A8       & 558.48 & 571.37 & 264.33 & 279.83 & 197.94 & 208.77 \\
		int paper 3A1       & 722.83 & 748.46 & 490.33 & 505.69 & 255.46 & 265.08 \\
		int paper B0        & 207.60 & 223.18 & 222.13 & 226.94 & 145.81 & 150.97
	\end{tabular}
\caption{Minimum and maximum box area values using our area finding algorithm}
\end{table}


\FloatBarrier


By manually entering the total number of chessboard squares, we guarantee that in each image, the chessboard squares are arranged in a quadrilateral shape. After we obtained the areas for each side of every box, we used the following calculations to help evaluate our algorithm's accuracy:\\ \\
1. \[ Error\ For\ Box\ Plane\ = \frac{\vert Defined\ Area - Test\ Result\ Area\vert}{Defined\ Area}  \]


2. \[Volume =Plane\ B\ avg*\sqrt{\frac{Plane\ C\ Avg* Plane\ A\ Avg}{Plane\ B\ Avg}} \]

3. \[ Volume\ Average\ Error\ = \frac{\vert Defined\ Volume\ - Test\ Result\ Volume\vert}{Defined\ Volume}  \]

\begin{figure} [H]


\begin{center}
	 \graphicspath{ {images/} }
	 \includegraphics[scale=.5]{volumechart.pdf}
%	\includepdf[pages={1}]{volumechart.pdf}
	\caption{Average volume results contrasted with pre=defined volumes}

\end{center}
\end{figure}

 
 Our results can also be viewed in the below column chart for easy comprehension. Our algorithm worked for the distorted and undistorted images. This result along with successful tests gave our area obtaining ratio validity. 
  \FloatBarrier

  
  \begin{figure}
  	\centering
  	\begin{minipage}{.5\textwidth}
  		\centering
  		\includegraphics[width=.4\linewidth]{1_A_2}
  		\\Box A3 Distorted Image: \\64.36 square inches 	
  	\end{minipage}%
  	\begin{minipage}{.5\textwidth}
  		\centering
  		\includegraphics[width=.4\linewidth]{1_A_4}
  		\\Box A3 Undistorted Image:\\ 63.73 square inches
  	\end{minipage}
  \end{figure}


 \begin{figure}
	\centering
	\begin{minipage}{.5\textwidth}
		\centering
		\includegraphics[width=.4\linewidth]{1_B_2}
		\\Box A3 Distorted Image: \\49.64 square inches 	
	\end{minipage}%
	\begin{minipage}{.5\textwidth}
		\centering
		\includegraphics[width=.4\linewidth]{1_B_3}
		\\Box A3 Undistorted Image:\\ 49.84 square inches
	\end{minipage}
\end{figure}


 \begin{figure}
	\centering
	\begin{minipage}{.5\textwidth}
		\centering
		\includegraphics[width=.4\linewidth]{3_A_4}
		\\10.25 x 6.75 x 3.25 Distorted Image: \\130.52 square inches
	\end{minipage}%
	\begin{minipage}{.5\textwidth}
		\centering
		\includegraphics[width=.4\linewidth]{3_A_3}
		\\10.25 x 6.75 x 3.25 Undistorted Image:\\  129.39 square inches 	
	\end{minipage}
\end{figure}

 
 \begin{figure}
	\centering
	\begin{minipage}{.5\textwidth}
		\centering
		\includegraphics[width=.4\linewidth]{5_A_5}
		\\Target Box Distorted Image: \\261.82 inches 	
	\end{minipage}%
	\begin{minipage}{.5\textwidth}
		\centering
		\includegraphics[width=.4\linewidth]{5_A_1}
		\\Target Box Undistorted Image:\\ 236.67 square inches
	\end{minipage}
\end{figure}



 \begin{figure}
	\centering
	\begin{minipage}{.5\textwidth}
		\centering
		\includegraphics[width=.4\linewidth]{9_B_3}
		\\Box A4 Distorted Image: \\76.57 square inches 	
	\end{minipage}%
	\begin{minipage}{.5\textwidth}
		\centering
		\includegraphics[width=.4\linewidth]{9_B_5}
		\\Box A4 Undistorted Image:\\ 74.59 square inches
	\end{minipage}
\end{figure}
\FloatBarrier

\subsection{Homography Results}

As discussed previously, we decided to use homographies to use as evidence that our proposed algorithm produces similar results. The homography results were very successful. Though we didn't test the homography program on all of our box images, the results are pretty much spot on with the pre-defined box areas.  \\
\FloatBarrier
\begin{table}[]
	\centering
	\caption{Plane A: True Values vs. Our Results vs. Homography, 2 photos per box side}
	\tiny
	\label{my-label}
	\begin{tabular}{lllll}
		Box                 & Our Test Result Avg (Plane A) & Plane A Defined Area & Plane A Area: Image 1 & Plane A Area: Image 2 \\
		A3 or BY1           & 66.91                         & 70.00                & 53.78                & 64.86                \\
		Target model 439 sm & 240.53                        & 236.25               & 201.73               & 192.10              
	\end{tabular}
\end{table}

\FloatBarrier
\begin{table}[]
	\centering
	\caption{Plane B: True Values vs. Our Results vs. Homography, 2 photos per box side}
	\tiny
	\label{my-label}
	\begin{tabular}{lllll}
		Box                 & Our Test Results (Plane B) & Plane B Defined Area & Plane B Area: Image 1 & Plane B Area: Image 2 \\
		A3 or BY1           & 51.26                      & 52.5                 & 39.69                & 41.37                \\
		Target model 439 sm & 153.07                     & 122.50               & 120.77               & 123.28              
	\end{tabular}
\end{table}
\FloatBarrier
\begin{table}[]
	\centering
	\tiny
	\caption{Plane C: True Values vs. Our Results vs. Homography, 2 photos per box side}
	\label{my-label}
	\begin{tabular}{lllll}
		Box                 & Our Test Results Avg (Plane C) & Plane C Defined Area & Plane C Area: Image 1 & Plane C Area: Image 2 \\
		A3 or BY1           & 42.79                          & 36.75                & 33.49                & 35.33                \\
		Target model 439 sm & 111.92                         & 94.50                & 94.54                & 93.48               
	\end{tabular}
\end{table}
\FloatBarrier

 In this diagram, the areas for the first box, plane A are not as accurate due to the fact that the areas in the images are not exactly flat as we were not consistent in taping down the box flaps. Though this doesn't necessarily justify our algorithm as being a method to find the volume of any box, the similar feedback between the homography and our proposed algorithm results is evidence that our method has the potential to be valid. 


\chapter{Discussion}
\label{label}

\section{Challenges/ Limitations} 

Over the course of this project, we faced a fair number of challenges and limitations that prevented certain aspects of the project from becoming reality. 
\\ \\
The first significant hurdle we encountered was the OpenCV \texttt{findChessboardCorners()} function. Though initially we aimed to use this function to automatically determine the number of small chessboard squares in each image, the function had too many problems in actually detecting the corners, that using it would make the results faulty. This function returned three types of erroneous feedback: \\



\begin{figure} [H]

		 \graphicspath{ {images/} }
		\caption{Detected no corners }

		\includegraphics[scale=.4]{none} 
	
\end{figure}

\begin{figure}[H]
 \graphicspath{ {images/} }
 \caption{Detected too few corners }

\includegraphics[scale=.5]{less} 
\end{figure}

\begin{figure}
	 \graphicspath{ {images/} }
	\caption{Detected too many corners and/ or in the wrong locations}
	\includegraphics[scale=.4]{too_many}
\end{figure}
 



Due to the temperamental nature of the \texttt{findChessboardCorners()} function, we resorted in having the user click on the outer four chessboard corners as well as the outer box area corners. This solution was not ideal but due to the time constriction, was necessary. 
\\ \\
Relying on the user to click the outer box and chessboard region corners showcased other limitations in our program. Not only does having the user click on the corners increase the likelihood of inaccuracies, but we learned that there are likely discrepancies between the initial box measurements (determined when the boxes are flat-packed in 2D) and when they are measured after being constructed in 3D. When a box is assembled, often the length, width, and height are amplified because of how a box is folded. Since the user clicks on the corners while the box is in 3D, this might have caused discrepancies in the outcome.\\ \\

 In addition, the data set of images could have been more precise. In many of the photos, the box flaps were not properly taped down so the chessboard pattern did not lay flat on the box side. This inconsistency made it so in many images the box sides weren't actually flat surfaces, altering the accuracy of the outcome. \\
 
 \begin{figure}
 	 \graphicspath{ {images/} }
 	
 	\includegraphics[scale=.05]{1_A_2} 
 	\caption{Box flaps are not properly taped}
 \end{figure}


Occasionally not all the box corners were displayed in the images. If we had access to a larger backdrop, it would have been easier to incorporate all the box corners into every frame. Also, with a larger backdrop, we would have been able to vary our test results more and use larger boxes, extending the distance between image points and the 3D points. \\
\begin{figure}[H]
	 \graphicspath{ {images/} }
		
		\includegraphics[scale=.05]{3_A_2} 
		\caption{Not all box corners are visible}
\end{figure}



In evaluating this project as a whole, it could have been more productive to control the angles and amounts of image distortion to obtain more consistent feedback. \\ \\
Also, we could have ensured more accuracy if we used all International Paper boxes that were pre-measured or only used boxes that were hand measured~\cite{intpaper}. This would have guaranteed more consistency in our data. 


\section{Conclusions} 


In conclusion, our results were more successful when images included more chessboard squares. 

\begin{figure}[H]
	 \graphicspath{ {images/} }

	 
	\includegraphics[scale=.05]{4_C_2} 
		 \caption{Less chessboard squares in photos result in decreased volume accuracy.}
\end{figure}


In addition, the fact that the homography results were similar to our algorithm's results suggests that our algorithm has the potential to be a valid way to determine a box's volume without the use of camera calibration. 

\chapter{Future Directions}

In terms of future pursuits, we can greatly improve this project in the following ways.\\

Incorporating the use of Harris corners could work better than the \texttt{findChessboardCorners()} function to automate the process of finding the number of squares.
Also, we could fully implement line detection to automatically detect the box areas' boundaries. It would be beneficial to explore the optimal amount of chessboard squares for a box surface. From a few experiments, we have found that the more chessboard squares on a box side, the more accurate the area calculation is of that surface. \\

\begin{figure}
	 \graphicspath{ {images/} }
	\includegraphics[scale=.2]{1A5_tape}
	\caption{Box with 246 chessboard squares}
\end{figure}
 


 In this example, the box side has a chessboard duct tape pattern on it. There are 246 small squares on this surface. The area of one small square is .0977 inches. When the area is calculated, the result is 134.14 square inches. Thus the error is only 9.7\% which is less than our average error for this side which was 14\%. Thus, this method of including more chessboard squares per surface shows preliminary potential. In addition, if we were to address the problem of millennials needing to know the total volume of all their boxes quickly, a more practical solution may be a mobile application platform for a future project. 

\chapter{Appendix}
\label{label}
\begin{centering}
	Finding Volume Algorithm Code
\end{centering}


\begin{Verbatim}

import Jama.*;
import gab.opencv.*;
import static javax.swing.JOptionPane.*;
import java.io.FileWriter;
import java.io.BufferedWriter;

float chessPixelArea;
ArrayList<PVector> cornerPoints;
OpenCV opencv;
int np1 = 0; 
PVector p1[] = new PVector[8];
int wid = 4;
int leng = 9;
int npMAX = 8; 
int num = 1;
int box_num;
PImage pix;

void setup() {
  background(255);
  smooth();

  pix = loadImage("box_6/side_A/6_A_2.jpg");

  pix.resize(500, 0);
  size(pix.width, pix.height);

  opencv = new OpenCV(this, pix);
  opencv.gray();

  cornerPoints = opencv.findChessboardCorners(leng, wid);
}

void mousePressed() {
  if (mouseX < width && mouseY <height) {
    p1[np1] = new PVector(mouseX, mouseY, 1);

    np1++;
  }
}

float calcPixelAreaREAL(PVector v1, PVector v2, PVector p, PVector p2) {

  PVector v1_copy = new PVector(v1.x, v1.y);
  PVector v2_copy = new PVector(v2.x, v2.y); 

  float a = v2.dist(p); 
  float b = v1.dist(p); 
  float c = p2.dist(v1); 
  float d = p2.dist(v2); 

  v1.sub(p);
  v2.sub(p);
  float p_ang = PVector.angleBetween(v1, v2);

  v1_copy.sub(p2); 
  v2_copy.sub(p2);

  float p2_ang = PVector.angleBetween(v1_copy, v2_copy);

  float areaa = .5*(a*b)*sin(p_ang) + .5*(c*d)*sin(p2_ang);
  return areaa;
}

void draw() { 

  image(opencv.getOutput(), 0, 0);


  noStroke();
  float chessRealArea = .828*box_num;
  for (PVector p : cornerPoints) {
    fill(255, 0, 0);
    ellipse(p.x, p.y, 5, 5);
  }
  if (np1 >= 8 && npMAX >=8) { 

    loadPixels(); 

    updatePixels();
  }

  for (int i = 0; i < np1; i++) {

    fill(0, 0, 255);
    ellipse( p1[i].x, p1[i].y, 20, 20);

    if (i==7) {

      PVector pv1 = new PVector (p1[0].x, p1[0].y);
      PVector v1 = new PVector (p1[1].x, p1[1].y);
      PVector pv2 = new PVector (p1[2].x, p1[2].y);
      PVector v2 = new PVector (p1[3].x, p1[3].y);

      PVector pp1 = new PVector(p1[4].x, p1[4].y);
      PVector vv1 = new PVector(p1[5].x, p1[5].y);
      PVector pp2 = new PVector(p1[6].x, p1[6].y);
      PVector vv2 = new PVector(p1[7].x, p1[7].y);

      chessPixelArea = calcPixelAreaREAL(v1, v2, pv1, pv2);
      float boxPixelArea = calcPixelAreaREAL(vv1, vv2, pp1, pp2);  

   
      try { 
        File f = dataFile("results_7.txt");  
        PrintWriter out = new PrintWriter(new BufferedWriter(new FileWriter(f, true)));
        out.println((boxPixelArea*chessRealArea)/chessPixelArea);
        out.flush();
        out.close();
      }
      catch (IOException e) {  
        println(e);
      }
      output.println("Test 5: Area of box plane is " + ((boxPixelArea*chessRealArea)/chessPixelArea) + " inch " + "box number:" +box_num);
      output.flush();  
      to the file
        output.close(); 
      exit();
    }
  }
}


..
\end{Verbatim}


\begin{Verbatim}
	import gab.opencv.*;
import org.opencv.imgproc.Imgproc;
import org.opencv.core.MatOfPoint2f;
import org.opencv.core.Point;
import org.opencv.core.Size;

import org.opencv.core.Mat;
import org.opencv.core.CvType;


OpenCV opencv;
PImage src;
float cardWidth = 3*.828;
float cardHeight = 8*.828;
int np1 = 0;
ArrayList<PVector> points = new ArrayList(4); 


void setup() {
  src = loadImage("5_C_5.jpg");
  src.resize(500, 500);
  size(src.width, src.height);
  opencv = new OpenCV(this, src);
}

void mousePressed() {
  if (mouseX < width && mouseY <height) {
    points.add(new PVector(mouseX, mouseY));
    println(mouseX, mouseY);
    np1++;
  }
}

Mat getPerspectiveTransformation(ArrayList<PVector> inputPoints, float w, float h) {
  Point[] canonicalPoints = new Point[4]; 
  canonicalPoints[0] = new Point(w, 0); 
  canonicalPoints[1] = new Point(0, 0);
  canonicalPoints[2] = new Point(0, h);
  canonicalPoints[3] = new Point(w, h);


  MatOfPoint2f canonicalMarker = new MatOfPoint2f();
  canonicalMarker.fromArray(canonicalPoints); 

  Point[] pointsArr = new Point[4]; 
  for (int i = 0; i < 4; i++) {

    pointsArr[i] = new Point(points.get(i).x, points.get(i).y);
  }
  MatOfPoint2f marker = new MatOfPoint2f();
  marker.fromArray(pointsArr);


  return Imgproc.getPerspectiveTransform(marker, canonicalMarker);
}


void draw() {
  image(src, 0, 0);

  for (int i = 0; i < np1; i++) {

    fill(255, 0, 0);
    ellipse( points.get(i).x, points.get(i).y, 20, 20);


    if (i==3) {


      Mat transform = getPerspectiveTransformation(points, cardWidth, cardHeight);

      println(transform.dump());
      noLoop();
      exit();
    }
  }
}


\end{Verbatim}

\begin{Verbatim}
	import gab.opencv.*;
import org.opencv.imgproc.Imgproc;
import org.opencv.core.MatOfPoint3f;
import org.opencv.core.Point;
import org.opencv.core.Size;
import org.opencv.core.Point3;

import org.opencv.core.Mat;
import org.opencv.core.CvType;
import org.opencv.core.Core;


OpenCV opencv;
PImage src;
float cardWidth = 3*.828;
float cardHeight = 8*.828;
int np1 = 0;
PVector pv1;
PVector v1;
PVector pv2;
PVector v2;



void setup() {
  src = loadImage("5_C_5.jpg");
  src.resize(500, 500);
  size(src.width, src.height);
  size(300, 300);
  opencv = new OpenCV(this, src);

  Point3[] canonicalPoints = new Point3[4]; 
  canonicalPoints[0] = new Point3(274, 151, 1); 
  canonicalPoints[1] = new Point3(110, 164, 1);
  canonicalPoints[2] = new Point3(139, 369, 1);
  canonicalPoints[3] = new Point3(312, 346, 1);


  Mat box_inch_pts = new Mat(3, 4, CvType.CV_32FC1); 

  Mat temp_H_inv = new Mat(3, 3, CvType.CV_32FC1); 
  Mat temp = new Mat(); 

  temp_H_inv.put(0, 0, new double[] { 

    0.05092402444659763, -0.005819888508182581, -7.062434704679569, 
    0.005086898655335161, 0.06994485651085894, -15.09282831037952, 
    0.0005803706726008797, 5.488099837932256e-05, 1
  }
  ); 
  

  box_inch_pts.put(0, 0, new double[] { 
    canonicalPoints[0].x, canonicalPoints[1].x, canonicalPoints[2].x, canonicalPoints[3].x, 
    canonicalPoints[0].y, canonicalPoints[1].y, canonicalPoints[2].y, canonicalPoints[3].y, 
    canonicalPoints[0].z, canonicalPoints[1].z, canonicalPoints[2].z, canonicalPoints[3].z
  }
  );

  
  Core.gemm(temp_H_inv, box_inch_pts, 1, new Mat(), 0, temp);


 

  Mat final_box_inch_pts = new Mat(3, 4, CvType.CV_32FC1); 
  
  //x's in nonhomogeneous
   println(temp.get(0, 0)[0]/temp.get(2, 0)[0]); //couner clockwise 
   println(temp.get(0, 1)[0]/temp.get(2, 1)[0]); //
   println(temp.get(0, 2)[0]/temp.get(2, 2)[0]);
   println(temp.get(0, 3)[0]/temp.get(2, 3)[0]);
   
   println();
   
   //y's in nonhomogeneous
   println(temp.get(1, 0)[0]/temp.get(2, 0)[0]); //couner clockwise 
   println(temp.get(1, 1)[0]/temp.get(2, 1)[0]); //
   println(temp.get(1, 2)[0]/temp.get(2, 2)[0]);
   println(temp.get(1, 3)[0]/temp.get(2, 3)[0]);
   
  

  pv1 = new PVector ((float)(temp.get(0, 1)[0]/temp.get(2, 1)[0]), (float)(temp.get(1, 1)[0]/temp.get(2, 1)[0]));
  v1 = new PVector (((float)temp.get(0, 0)[0]/(float)temp.get(2, 0)[0]), (float)(temp.get(1, 0)[0]/temp.get(2, 0)[0]));
  pv2 = new PVector((float)(temp.get(0, 3)[0]/temp.get(2, 3)[0]), (float)(temp.get(1, 3)[0]/temp.get(2, 3)[0]));
  v2 = new PVector ((float)(temp.get(0, 2)[0]/temp.get(2, 2)[0]), (float)(temp.get(1, 2)[0]/temp.get(2, 2)[0]));
  /*
  final_box_inch_pts.put(0, 0, new double[] { 
   (float)temp.get(0, 0)/(float)temp.get(2, 0), temp.get(0, 1)/temp.get(2, 1), temp.get(0, 2)/temp.get(2, 2), temp.get(0, 3)/temp.get(2, 3), 
   temp.get(1, 0)/temp.get(2, 0), temp.get(1, 1)/temp.get(2, 1), temp.get(1, 2)/temp.get(2, 2), temp.get(1, 3)/temp.get(2, 3), 
   temp.get(2, 0)/temp.get(2, 0), temp.get(2, 1)/temp.get(2, 1), temp.get(2, 2)/temp.get(2, 2), temp.get(2, 3)/temp.get(2, 3)
   }
   ); */


}



float calcPixelAreaREAL(PVector v1, PVector v2, PVector p, PVector p2) {

  PVector v1_copy = new PVector(v1.x, v1.y);
  PVector v2_copy = new PVector(v2.x, v2.y); 

  float a = v2.dist(p); //v2.y - p.y;
  float b = v1.dist(p); //v1.x - p.x;
  float c = p2.dist(v1); //p2.y-v1.y;
  float d = p2.dist(v2); ///p2.x - v2.x;

  v1.sub(p);
  v2.sub(p);
  float p_ang = PVector.angleBetween(v1, v2);
  

  v1_copy.sub(p2); 
  v2_copy.sub(p2);

  float p2_ang = PVector.angleBetween(v1_copy, v2_copy);


  float areaa = .5*(a*b)*sin(p_ang) + .5*(c*d)*sin(p2_ang);

  return areaa;


}






void draw() {
  float box_real_area = calcPixelAreaREAL(v1, v2, pv1, pv2);
  println("box real area is " + box_real_area);
  exit();
}


\end{Verbatim}


 
\FloatBarrier

\begin{table}[]
	\centering

	\label{AR1A}
	\begin{tabular}{lllll}
		Box                 & length & width & height & Defined Area (A) \\
		A3 or BY1           & 10     & 7     & 5.25   & 70               \\
		Shar                & 10.25  & 6.75  & 3.25   & 69.19            \\
		int paper 1AD BNA   & 13.5   & 9.5   & 3.5    & 128.25           \\
		LLBean              & 13.75  & 8     & 5      & 110              \\
		Target model 439 sm & 17.5   & 13.5  & 7      & 236.25           \\
		N3 or B41           & 16.25  & 12    & 5.25   & 195              \\
		CVS box             & 21.65  & 13.4  & 12.6   & 290.11           \\
		Lowes Box           & 16     & 12    & 12     & 192              \\
		Int Paper 1A5 B45   & 13.5   & 11    & 4.75   & 148.5            \\
		int paper 0A0       & 9.25   & 6.5   & 2.25   & 60.13            \\
		Int paper A4        & 12     & 8.5   & 7      & 102              \\
		Int paper 1A7       & 14.5   & 8     & 7.25   & 116              \\
		int paper BP0       & 10     & 8     & 2      & 80               \\
		int paper 2AA       & 24     & 16    & 4.25   & 384              \\
		int paper 2A0       & 20     & 16    & 7      & 320              \\
		int paper BP1       & 13.25  & 9.5   & 2.25   & 125.88           \\
		int paper 2A5       & 20     & 11    & 4.25   & 220              \\
		int paper 2A8       & 26     & 19    & 9      & 494              \\
		int paper 3A1       & 31     & 20.5  & 10.5   & 635.5            \\
		int paper B0        & 17     & 11    & 11     & 187             
	\end{tabular}
	\caption{Defined values for tested boxes}
\end{table}
\FloatBarrier

\begin{table}[]
	\centering
	
	\label{AR1B}
	\tiny
	\begin{tabular}{llllllll}
		Box                 & Defined Area (B) & Defined Area (C) & test 1(A) & test 2 (A) & test 3 (A) & test 4 (A) & test 5 (A) \\
		A3 or BY1           & 52.50            & 36.75            & 79.30     & 64.36      & 63.65      & 63.73      & 63.49      \\
		Shar                & 33.31            & 21.94            & 66.99     & 71.52      & 66.38      & 69.53      & 68.65      \\
		int paper 1AD BNA   & 47.25            & 33.25            & 114.11    & 107.46     & 129.39     & 130.52     & 127.10     \\
		LLBean              & 68.75            & 40.00            & 100.72    & 106.58     & 106.30     & 106.21     & 109.20     \\
		Target model 439 sm & 122.50           & 94.50            & 236.67    & 223.35     & 247.95     & 232.86     & 261.82     \\
		N3 or B41           & 85.31            & 63.00            & 186.28    & 191.83     & 196.50     & 187.43     & 189.50     \\
		CVS box             & 272.79           & 168.84           & 304.77    & 288.95     & 274.52     & 269.46     & 273.30     \\
		Lowes Box           & 192.00           & 144.00           & 206.66    & 203.69     & 210.26     & 206.09     & 212.54     \\
		Int Papr 1A5 B45    & 64.13            & 52.25            & 167.55    & 168.69     & 171.49     & 172.50     & 163.07     \\
		int paper 0A0       & 20.81            & 14.63            & 52.17     & 53.74      & 58.90      & 60.53      & 59.70      \\
		Int paper A4        & 84.00            & 59.50            & 116.07    & 114.49     & 113.51     & 115.65     & 114.23     \\
		Int paper 1A7       & 105.13           & 58.00            & 142.20    & 143.18     & 141.02     & 140.13     & 137.09     \\
		int paper BP0       & 20.00            & 16.00            & 90.58     & 91.85      & 89.49      & 88.00      & 95.74      \\
		int paper 2AA       & 102.00           & 68.00            & 439.17    & 441.32     & 437.66     & 441.88     & 445.27     \\
		int paper 2A0       & 140.00           & 112.00           & 370.94    & 370.17     & 374.53     & 374.02     & 373.25     \\
		int paper BP1       & 29.81            & 21.38            & 148.78    & 145.03     & 142.35     & 149.43     & 142.98     \\
		int paper 2A5       & 85.00            & 46.75            & 245.68    & 245.55     & 244.66     & 251.56     & 251.45     \\
		int paper 2A8       & 234.00           & 171.00           & 565.59    & 564.16     & 558.48     & 568.03     & 571.37     \\
		int paper 3A1       & 325.50           & 215.25           & 742.18    & 748.46     & 725.30     & 722.83     & 731.71     \\
		int paper B0        & 187.00           & 121.00           & 215.34    & 213.22     & 211.93     & 207.60     & 223.18    
	\end{tabular}
\caption{Box side A area results compared with pre-determined values}
\end{table}

\begin{table}[]
	\centering

	\label{AR1C}
	\small
	\begin{tabular}{llllll}
		Box                 & test 1(B) & test 2 (B) & test 3 (B) & test 4 (B) & test 5 (B) \\
		A3 or BY1           & 50.29     & 49.64      & 49.84      & 52.71      & 53.81      \\
		Shar                & 36.34     & 36.35      & 38.36      & 36.55      & 36.78      \\
		int paper 1AD BNA   & 40.66     & 40.45      & 42.47      & 40.29      & 42.21      \\
		LLBean              & 62.11     & 58.39      & 63.74      & 57.61      & 60.22      \\
		Target model 439 sm & 144.53    & 151.60     & 167.32     & 153.33     & 148.57     \\
		N3 or B41           & 85.07     & 75.23      & 96.82      & 74.32      & 75.90      \\
		CVS box             & 256.04    & 264.88     & 276.51     & 260.95     & 297.82     \\
		Lowes Box           & 206.95    & 212.55     & 206.06     & 204.79     & 205.55     \\
		Int Paper 1A5 B45   & 78.10     & 78.97      & 76.57      & 76.10      & 74.59      \\
		int paper 0A0       & 20.46     & 21.00      & 20.46      & 21.58      & 22.36      \\
		Int paper A4        & 87.01     & 84.95      & 82.99      & 89.00      & 93.98      \\
		Int paper 1A7       & 115.64    & 114.72     & 113.98     & 113.87     & 112.46     \\
		int paper BP0       & 19.78     & 19.49      & 20.61      & 19.34      & 20.94      \\
		int paper 2AA       & 112.80    & 113.98     & 115.18     & 109.66     & 113.65     \\
		int paper 2A0       & 155.24    & 157.30     & 154.08     & 152.56     & 159.88     \\
		int paper BP1       & 32.16     & 32.60      & 32.16      & 31.10      & 33.45      \\
		int paper 2A5       & 98.47     & 94.54      & 96.65      & 97.12      & 102.34     \\
		int paper 2A8       & 271.04    & 268.11     & 264.33     & 278.41     & 279.83     \\
		int paper 3A1       & 490.63    & 494.42     & 496.68     & 505.69     & 490.33     \\
		int paper B0        & 222.13    & 224.38     & 224.36     & 226.94     & 222.31    
	\end{tabular}
\caption{Box side B area results for each test}
\end{table}


\begin{table}[]
	\centering

	\label{AR1D}
	\begin{tabular}{llllll}
		Box                 & test 1(C) & test 2 (C) & test 3 (C) & test 4 (C) & test 5 (C) \\
		A3 or BY1           & 39.20     & 43.96      & 42.43      & 49.57      & 38.78      \\
		Shar                & 23.40     & 19.04      & 20.54      & 17.78      & 18.74      \\
		int paper 1AD BNA   & 32.55     & 33.63      & 36.33      & 33.72      & 32.35      \\
		LLBean              & 37.48     & 36.07      & 34.94      & 38.92      & 36.32      \\
		Target model 439 sm & 103.13    & 113.92     & 117.01     & 115.50     & 110.06     \\
		N3 or B41           & 63.71     & 65.29      & 63.53      & 35.64      & 36.18      \\
		CVS box             & 160.66    & 163.71     & 160.35     & 152.08     & 159.56     \\
		Lowes Box           & 168.25    & 161.57     & 159.21     & 166.84     & 174.29     \\
		Int Paper 1A5 B45   & 58.32     & 59.75      & 61.04      & 56.75      & 58.33      \\
		int paper 0A0       & 16.43     & 14.18      & 15.45      & 16.76      & 16.78      \\
		Int paper A4        & 64.93     & 67.22      & 64.42      & 63.55      & 63.58      \\
		Int paper 1A7       & 63.91     & 66.11      & 67.08      & 68.03      & 64.93      \\
		int paper BP0       & 18.49     & 18.21      & 19.25      & 15.92      & 16.72      \\
		int paper 2AA       & 77.77     & 79.06      & 79.83      & 82.50      & 76.97      \\
		int paper 2A0       & 129.41    & 127.12     & 127.84     & 133.15     & 131.49     \\
		int paper BP1       & 22.64     & 23.35      & 21.92      & 23.81      & 25.79      \\
		int paper 2A5       & 51.91     & 51.74      & 55.73      & 56.71      & 56.89      \\
		int paper 2A8       & 208.77    & 197.94     & 203.72     & 201.28     & 206.76     \\
		int paper 3A1       & 256.71    & 255.46     & 265.08     & 264.22     & 264.44     \\
		int paper B0        & 147.43    & 148.27     & 150.97     & 145.81     & 148.38    
	\end{tabular}
	\caption{Box side C area results for each test}
\end{table}

\begin{table}[]
	\centering
	
	\label{AR1El}
	\tiny
	\begin{tabular}{llllll}
		Box                 & (A) Error Test 1 & (A) Error Test 2 & (A) Error Test 3 & (A) Error Test 4 & (A) Error Test 5 \\
		A3 or BY1           & 13.29\%          & 8.05\%           & 9.07\%           & 8.96\%           & 9.29\%           \\
		Shar                & 3.18\%           & 3.37\%           & 4.05\%           & 0.50\%           & 0.77\%           \\
		int paper 1AD BNA   & 11.02\%          & 16.21\%          & 0.89\%           & 1.77\%           & 0.89\%           \\
		LLBean              & 8.44\%           & 3.11\%           & 3.36\%           & 3.45\%           & 0.73\%           \\
		Target model 439 sm & 0.18\%           & 5.46\%           & 4.95\%           & 1.43\%           & 10.82\%          \\
		N3 or B41           & 4.47\%           & 1.63\%           & 0.77\%           & 3.88\%           & 2.82\%           \\
		CVS box             & 5.05\%           & 0.40\%           & 5.37\%           & 7.12\%           & 5.80\%           \\
		Lowes Box           & 7.63\%           & 6.09\%           & 9.51\%           & 7.34\%           & 10.70\%          \\
		Int Paper 1A5 B45   & 12.83\%          & 13.60\%          & 15.48\%          & 16.16\%          & 9.81\%           \\
		int paper 0A0       & 13.23\%          & 10.62\%          & 2.04\%           & 0.67\%           & 0.70\%           \\
		Int paper A4        & 13.80\%          & 12.24\%          & 11.28\%          & 13.38\%          & 11.99\%          \\
		Int paper 1A7       & 22.58\%          & 23.43\%          & 21.57\%          & 20.80\%          & 18.18\%          \\
		int paper BP0       & 13.23\%          & 14.81\%          & 11.86\%          & 10.00\%          & 19.68\%          \\
		int paper 2AA       & 14.37\%          & 14.93\%          & 13.97\%          & 15.07\%          & 15.96\%          \\
		int paper 2A0       & 15.92\%          & 15.68\%          & 17.04\%          & 16.88\%          & 16.64\%          \\
		int paper BP1       & 18.19\%          & 15.22\%          & 13.09\%          & 18.72\%          & 13.59\%          \\
		int paper 2A5       & 11.67\%          & 11.61\%          & 11.21\%          & 14.35\%          & 14.29\%          \\
		int paper 2A8       & 14.49\%          & 14.20\%          & 13.05\%          & 14.98\%          & 15.66\%          \\
		int paper 3A1       & 16.79\%          & 17.77\%          & 14.13\%          & 13.74\%          & 15.14\%          \\
		int paper B0        & 15.15\%          & 14.02\%          & 13.33\%          & 11.02\%          & 19.35\%         
	\end{tabular}
\caption{Box side A error percentages}
\end{table}


\begin{table}[]
	\centering

	\label{AR1F}
	\tiny
	\begin{tabular}{llllll}
		Box                 & (B) Error Test 1 & (B) Error Test 2 & (B) Error Test 3 & (B) Error Test 4 & (B) Error Test 5 \\
		A3 or BY1           & 4.21\%           & 5.45\%           & 5.06\%           & 0.40\%           & 2.50\%           \\
		Shar                & 9.09\%           & 9.12\%           & 15.17\%          & 9.71\%           & 10.40\%          \\
		int paper 1AD BNA   & 13.95\%          & 14.38\%          & 10.12\%          & 14.74\%          & 10.67\%          \\
		LLBean              & 9.66\%           & 15.07\%          & 7.29\%           & 16.20\%          & 12.41\%          \\
		Target model 439 sm & 17.98\%          & 23.75\%          & 36.59\%          & 25.17\%          & 21.28\%          \\
		N3 or B41           & 0.28\%           & 11.82\%          & 13.49\%          & 12.88\%          & 11.03\%          \\
		CVS box             & 6.14\%           & 2.90\%           & 1.36\%           & 4.34\%           & 9.17\%           \\
		Lowes Box           & 7.79\%           & 10.70\%          & 7.33\%           & 6.66\%           & 7.06\%           \\
		Int Paper 1A5 B45   & 21.79\%          & 23.16\%          & 19.41\%          & 18.67\%          & 16.32\%          \\
		int paper 0A0       & 1.70\%           & 0.91\%           & 1.70\%           & 3.70\%           & 7.41\%           \\
		Int paper A4        & 3.58\%           & 1.13\%           & 1.21\%           & 5.95\%           & 11.88\%          \\
		Int paper 1A7       & 10.00\%          & 9.13\%           & 8.43\%           & 8.32\%           & 6.98\%           \\
		int paper BP0       & 1.10\%           & 2.54\%           & 3.07\%           & 3.31\%           & 4.68\%           \\
		int paper 2AA       & 10.59\%          & 11.75\%          & 12.93\%          & 7.51\%           & 11.42\%          \\
		int paper 2A0       & 10.88\%          & 12.36\%          & 10.06\%          & 8.97\%           & 14.20\%          \\
		int paper BP1       & 7.87\%           & 9.34\%           & 7.87\%           & 4.33\%           & 12.20\%          \\
		int paper 2A5       & 15.84\%          & 11.23\%          & 13.70\%          & 14.26\%          & 20.39\%          \\
		int paper 2A8       & 15.83\%          & 14.58\%          & 12.96\%          & 18.98\%          & 19.58\%          \\
		int paper 3A1       & 50.73\%          & 51.90\%          & 52.59\%          & 55.36\%          & 50.64\%          \\
		int paper B0        & 18.78\%          & 19.99\%          & 19.98\%          & 21.36\%          & 18.88\%         
	\end{tabular}
\caption{Box side B error percentages}
\end{table}


\begin{table}[]
	\centering

	\label{AR1G}
	\tiny
	\begin{tabular}{llllll}
		Box                 & (C) Error Test 1 & (C) Error Test 2 & (C) Error Test 3 & (C) Error Test 4 & (C) Error Test 5 \\
		A3 or BY1           & 6.68\%           & 19.61\%          & 15.45\%          & 34.87\%          & 5.53\%           \\
		Shar                & 6.67\%           & 13.21\%          & 6.38\%           & 18.97\%          & 14.56\%          \\
		int paper 1AD BNA   & 2.12\%           & 1.14\%           & 9.25\%           & 1.42\%           & 2.70\%           \\
		LLBean              & 6.30\%           & 9.82\%           & 12.65\%          & 2.70\%           & 9.21\%           \\
		Target model 439 sm & 9.13\%           & 20.56\%          & 23.82\%          & 22.22\%          & 16.46\%          \\
		N3 or B41           & 1.13\%           & 3.63\%           & 0.84\%           & 43.42\%          & 42.58\%          \\
		CVS box             & 4.84\%           & 3.04\%           & 5.03\%           & 9.93\%           & 5.50\%           \\
		Lowes Box           & 16.84\%          & 12.20\%          & 10.56\%          & 15.86\%          & 21.04\%          \\
		Int Paper 1A5 B45   & 11.62\%          & 14.36\%          & 16.83\%          & 8.61\%           & 11.64\%          \\
		int paper 0A0       & 12.37\%          & 3.06\%           & 5.67\%           & 14.58\%          & 14.75\%          \\
		Int paper A4        & 9.12\%           & 12.98\%          & 8.27\%           & 6.80\%           & 6.86\%           \\
		Int paper 1A7       & 10.18\%          & 13.99\%          & 15.66\%          & 17.30\%          & 11.94\%          \\
		int paper BP0       & 15.54\%          & 13.84\%          & 20.29\%          & 0.51\%           & 4.49\%           \\
		int paper 2AA       & 14.37\%          & 16.27\%          & 17.39\%          & 21.32\%          & 13.20\%          \\
		int paper 2A0       & 15.55\%          & 13.50\%          & 14.14\%          & 18.89\%          & 17.40\%          \\
		int paper BP1       & 5.92\%           & 9.24\%           & 2.54\%           & 11.41\%          & 20.64\%          \\
		int paper 2A5       & 11.04\%          & 10.68\%          & 19.20\%          & 21.30\%          & 21.69\%          \\
		int paper 2A8       & 22.09\%          & 15.75\%          & 19.13\%          & 17.71\%          & 20.91\%          \\
		int paper 3A1       & 19.26\%          & 18.68\%          & 23.15\%          & 22.75\%          & 22.85\%          \\
		int paper B0        & 21.85\%          & 22.54\%          & 24.77\%          & 20.50\%          & 22.62\%         
	\end{tabular}
\caption{Box side C error percentages}
\end{table}







\begin{table}[]
	\centering

	\label{AR2A}
	\begin{tabular}{llll}
		Box                 & (A) Avg & (B) Avg & (C) Avg \\
		A3 or BY1           & 66.91   & 51.26   & 42.79   \\
		Shar                & 68.62   & 36.88   & 19.90   \\
		int paper 1AD BNA   & 121.72  & 41.22   & 33.72   \\
		LLBean              & 105.80  & 60.41   & 36.75   \\
		Target model 439 sm & 240.53  & 153.07  & 111.92  \\
		N3 or B41           & 190.31  & 81.47   & 52.87   \\
		CVS box             & 282.20  & 271.24  & 159.27  \\
		Lowes Box           & 207.85  & 207.18  & 166.03  \\
		Int Paper 1A5 B45   & 168.66  & 76.87   & 58.84   \\
		int paper 0A0       & 57.01   & 21.17   & 15.92   \\
		Int paper A4        & 114.79  & 87.58   & 64.74   \\
		Int paper 1A7       & 140.72  & 114.13  & 66.01   \\
		int paper BP0       & 91.13   & 20.03   & 17.72   \\
		int paper 2AA       & 441.06  & 113.06  & 79.23   \\
		int paper 2A0       & 372.58  & 155.81  & 129.80  \\
		int paper BP1       & 145.72  & 32.29   & 23.50   \\
		int paper 2A5       & 247.78  & 97.82   & 54.60   \\
		int paper 2A8       & 565.52  & 272.34  & 203.69  \\
		int paper 3A1       & 734.10  & 495.55  & 261.18  \\
		int paper B0        & 214.26  & 224.02  & 148.17 
	\end{tabular}
\caption{Average area for each box side}
\end{table}

\begin{table}[]
	\centering

	\label{AR2B}
	\tiny
	\begin{tabular}{lllllll}
		Box                 & (A) Min Error & (B) Min error & (C) Min Error & (A) Max Error & (B) Max Error & (C) Max Error \\
		A3 or BY1           & 8.05\%        & 0.40\%        & 5.53\%        & 13.29\%       & 5.45\%        & 34.87\%       \\
		Shar                & 0.50\%        & 9.09\%        & 6.38\%        & 4.05\%        & 15.17\%       & 18.97\%       \\
		int paper 1AD BNA   & 0.89\%        & 10.12\%       & 1.14\%        & 16.21\%       & 14.74\%       & 9.25\%        \\
		LLBean              & 0.73\%        & 7.29\%        & 2.70\%        & 8.44\%        & 16.20\%       & 12.65\%       \\
		Target model 439 sm & 0.18\%        & 17.98\%       & 9.13\%        & 10.82\%       & 36.59\%       & 23.82\%       \\
		N3 or B41           & 0.77\%        & 0.28\%        & 0.84\%        & 4.47\%        & 13.49\%       & 43.42\%       \\
		CVS box             & 0.40\%        & 1.36\%        & 3.04\%        & 7.12\%        & 9.17\%        & 9.93\%        \\
		Lowes Box           & 6.09\%        & 6.66\%        & 10.56\%       & 10.70\%       & 10.70\%       & 21.04\%       \\
		Int Paper 1A5 B45   & 9.81\%        & 16.32\%       & 8.61\%        & 16.16\%       & 23.16\%       & 16.83\%       \\
		int paper 0A0       & 0.67\%        & 0.91\%        & 3.06\%        & 13.23\%       & 7.41\%        & 14.75\%       \\
		Int paper A4        & 11.28\%       & 1.13\%        & 6.80\%        & 13.80\%       & 11.88\%       & 12.98\%       \\
		Int paper 1A7       & 18.18\%       & 6.98\%        & 10.18\%       & 23.43\%       & 10.00\%       & 17.30\%       \\
		int paper BP0       & 10.00\%       & 1.10\%        & 0.51\%        & 19.68\%       & 4.68\%        & 20.29\%       \\
		int paper 2AA       & 13.97\%       & 7.51\%        & 13.20\%       & 15.96\%       & 12.93\%       & 21.32\%       \\
		int paper 2A0       & 15.68\%       & 8.97\%        & 13.50\%       & 17.04\%       & 14.20\%       & 18.89\%       \\
		int paper BP1       & 13.09\%       & 4.33\%        & 2.54\%        & 18.72\%       & 12.20\%       & 20.64\%       \\
		int paper 2A5       & 11.21\%       & 11.23\%       & 10.68\%       & 14.35\%       & 20.39\%       & 21.69\%       \\
		int paper 2A8       & 13.05\%       & 12.96\%       & 15.75\%       & 15.66\%       & 19.58\%       & 22.09\%       \\
		int paper 3A1       & 13.74\%       & 50.64\%       & 18.68\%       & 17.77\%       & 55.36\%       & 23.15\%       \\
		int paper B0        & 11.02\%       & 18.78\%       & 20.50\%       & 19.35\%       & 21.36\%       & 24.77\%      
	\end{tabular}
\caption{Min and max error for each box side}
\end{table}


\begin{table}[]
	\centering

	\label{AR2C}
	\begin{tabular}{llll}
		Box                 & (A) Avg Error & (B) Avg Error & (C) Avg Error \\
		A3 or BY1           & 4.42\%        & 2.36\%        & 16.43\%       \\
		Shar                & 0.83\%        & 10.70\%       & 9.29\%        \\
		int paper 1AD BNA   & 5.09\%        & 12.77\%       & 1.40\%        \\
		LLBean              & 3.82\%        & 12.13\%       & 8.14\%        \\
		Target model 439 sm & 1.81\%        & 24.95\%       & 18.44\%       \\
		N3 or B41           & 2.41\%        & 4.51\%        & 16.08\%       \\
		CVS box             & 2.73\%        & 0.57\%        & 5.67\%        \\
		Lowes Box           & 8.25\%        & 7.91\%        & 15.30\%       \\
		Int Paper 1A5 B45   & 13.58\%       & 19.87\%       & 12.61\%       \\
		int paper 0A0       & 5.18\%        & 1.72\%        & 8.86\%        \\
		Int paper A4        & 12.54\%       & 4.27\%        & 8.81\%        \\
		Int paper 1A7       & 21.31\%       & 8.57\%        & 13.81\%       \\
		int paper BP0       & 13.92\%       & 0.16\%        & 10.73\%       \\
		int paper 2AA       & 14.86\%       & 10.84\%       & 16.51\%       \\
		int paper 2A0       & 16.43\%       & 11.29\%       & 15.89\%       \\
		int paper BP1       & 15.76\%       & 8.32\%        & 9.95\%        \\
		int paper 2A5       & 12.63\%       & 15.09\%       & 16.78\%       \\
		int paper 2A8       & 14.48\%       & 16.39\%       & 19.12\%       \\
		int paper 3A1       & 15.51\%       & 52.24\%       & 21.34\%       \\
		int paper B0        & 14.58\%       & 19.80\%       & 22.46\%      
	\end{tabular}
\caption{Average error for each box side}
\end{table}


\begin{table}[]
	\centering

	\label{AR2Dl}
	\begin{tabular}{lllllll}
		Box                 & Min A  & Max A  & Min B  & Max B  & Min C  & Max C  \\
		A3 or BY1           & 63.49  & 79.30  & 49.64  & 53.81  & 38.78  & 49.57  \\
		Shar                & 66.38  & 71.52  & 36.34  & 38.36  & 17.78  & 23.40  \\
		int paper 1AD BNA   & 107.46 & 130.52 & 40.29  & 42.47  & 32.35  & 36.33  \\
		LLBean              & 100.72 & 109.20 & 57.61  & 63.74  & 34.94  & 38.92  \\
		Target model 439 sm & 223.35 & 261.82 & 144.53 & 167.32 & 103.13 & 117.01 \\
		N3 or B41           & 186.28 & 196.50 & 74.32  & 96.82  & 35.64  & 65.29  \\
		CVS box             & 269.46 & 304.77 & 256.04 & 297.82 & 152.08 & 163.71 \\
		Lowes Box           & 203.69 & 212.54 & 204.79 & 212.55 & 159.21 & 174.29 \\
		Int Paper 1A5 B45   & 163.07 & 172.50 & 74.59  & 78.97  & 56.75  & 61.04  \\
		int paper 0A0       & 52.17  & 60.53  & 20.46  & 22.36  & 14.18  & 16.78  \\
		Int paper A4        & 113.51 & 116.07 & 82.99  & 93.98  & 63.55  & 67.22  \\
		Int paper 1A7       & 137.09 & 143.18 & 112.46 & 115.64 & 63.91  & 68.03  \\
		int paper BP0       & 88.00  & 95.74  & 19.34  & 20.94  & 15.92  & 19.25  \\
		int paper 2AA       & 437.66 & 445.27 & 109.66 & 115.18 & 76.97  & 82.50  \\
		int paper 2A0       & 370.17 & 374.53 & 152.56 & 159.88 & 127.12 & 133.15 \\
		int paper BP1       & 142.35 & 149.43 & 31.10  & 33.45  & 21.92  & 25.79  \\
		int paper 2A5       & 244.66 & 251.56 & 94.54  & 102.34 & 51.74  & 56.89  \\
		int paper 2A8       & 558.48 & 571.37 & 264.33 & 279.83 & 197.94 & 208.77 \\
		int paper 3A1       & 722.83 & 748.46 & 490.33 & 505.69 & 255.46 & 265.08 \\
		int paper B0        & 207.60 & 223.18 & 222.13 & 226.94 & 145.81 & 150.97
	\end{tabular}
\caption{Min and max areas for each box side}
\end{table}

\begin{table}[]
	\centering

	\label{AR2E}
	\begin{tabular}{llll}
		Box                 & Avg Volume & Defined Volume & Volume Avg Error \\
		A3 or BY1           & 383.07     & 367.50         & 4.24\%           \\
		Shar                & 153.15     & 224.86         & 31.89\%          \\
		int paper 1AD BNA   & 280.70     & 448.88         & 37.47\%          \\
		LLBean              & 330.77     & 550.00         & 39.86\%          \\
		Target model 439 sm & 1385.50    & 1653.75        & 16.22\%          \\
		N3 or B41           & 617.94     & 1023.75        & 39.64\%          \\
		CVS box             & 2383.09    & 3655.39        & 34.81\%          \\
		Lowes Box           & 1824.99    & 2304.00        & 20.79\%          \\
		Int Paper 1A5 B45   & 596.10     & 705.38         & 15.49\%          \\
		int paper 0A0       & 94.61      & 135.28         & 30.06\%          \\
		Int paper A4        & 550.63     & 714.00         & 22.88\%          \\
		Int paper 1A7       & 702.78     & 841.00         & 16.44\%          \\
		int paper BP0       & 122.75     & 160.00         & 23.28\%          \\
		int paper 2AA       & 1356.58    & 1632.00        & 16.88\%          \\
		int paper 2A0       & 1873.55    & 2240.00        & 16.36\%          \\
		int paper BP1       & 226.97     & 283.22         & 19.86\%          \\
		int paper 2A5       & 785.14     & 935.00         & 16.03\%          \\
		int paper 2A8       & 3822.86    & 4446.00        & 14.02\%          \\
		int paper 3A1       & 6652.82    & 6672.75        & 0.30\%           \\
		int paper B0        & 1820.16    & 2057.00        & 11.51\%         
	\end{tabular}
\caption{Volume results}
\end{table}


\begin{bibliog}
	
\bib {sze}{book}{
		author = {Szeliski, Richard},
		title = {Computer Vision: Algorithms and Applications},
		year = {2010},
		isbn = {1848829345, 9781848829343},
		edition = {1st},
		publisher = {Springer-Verlag New York, Inc.},
		address = {New York, NY, USA},
	}

\bib{Solem}{book}{
	author = {Solem, Jan Erik},
	title = {Programming Computer Vision with Python},
	publisher = {O'Reilly},
%	address = {city},
	date = {2012}
}



 \bib{moving}{article}{
 	author={Godfrey, Neale}, 
 	title={The Young And The Restless: Millennials On The Move},
 	journal={Forbes},
 
 	
 	publisher={Forbes Magazine}, 
 	 
 	date={2016}}
 
\bib{gregProc}{article}{
	author = {Atduskgreg},
	title = {atduskgreg/opencv-processing},
%	journal = {Github},
	%volume = {volume number},
	date = {2017}
%	pages = {starting page--ending page}
}
\bib{opencvCC}{article}{
	%author = {last name, first name},
	title = {Camera calibration With OpenCV — OpenCV 2.4.13.4 documentation},
	journal = {OpenCV},
	%volume = {volume number},
	date = {2017}
	%	pages = {starting page--ending page}
}

\bib{pyopencv}{article}{
	author = {Rosebrock,Adrian},
	title = {Find distance from camera to object using Python and OpenCV},
	journal = {Py Image Search},
	%volume = {volume number},
	date = {July 12, 2016}
	%pages = {starting page--ending page}
}
	
\bib{census}{article}{

		author = {Benetsky, Megan J.},
		author = {Burd, Charlynn A.},
		author = {Rapino, Melanie A.},
		title = {Young Adult Migration: 2007–2009 to 2010–2012},
		journal = {Census.gov},
		%volume = {volume number},
		date = {March 2015}
	%	pages = {starting page--ending page}
}

\bib{quadArea}{article}{
	%author = {last name, first name},
	title = {How to Find the Area of a Quadrilateral},
	journal = {wikiHow},
	%volume = {volume number},
	date = {May 22, 2017}
	%pages = {starting page--ending page}
}
\bib{intpaper}{article}{
	author = {N\/A, Kevin},
	title = {A Catalog of Amazon.com Box Sizes},
	journal = {Incomptech},
%	volume = {volume number},
%	date = {year}
%	pages = {starting page--ending page}
}


\bib{spixel}{article}{
%	author = {last name, first name},
	title = {3D Viewing: the Pinhole Camera Model},
	journal = {Scratch Pixel},
	%volume = {volume number},
	%date = {year}
%	pages = {starting page--ending page}
}

\bib{homography}{article}{

	author = {Taubin, Gabriel},
	author = {Moreno, Daniel},
	title = {Simple, Accurate, and Robust Projector-Camera Calibration},
	journal = {Brown University School of Engineering},
	%volume = {volume number},
	date = {2012}
	%pages = {starting page--ending page}
}






\end{bibliog}

\end{document}

